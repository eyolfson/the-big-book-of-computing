% Copyright 2015 Jonathan Eyolfson
%
% This work is licensed under a Creative Commons Attribution-ShareAlike 4.0
% International License. You should have received a copy of the license along
% with this work. If not, see <http://creativecommons.org/licenses/by-sa/4.0/>.

\documentclass[12pt]{book}

\usepackage{amsmath}
\usepackage{fontspec}
\usepackage[margin=2.54cm, paper=letterpaper]{geometry}
\usepackage[newfloat]{minted}
\usepackage{tikz}

\setmainfont{Tinos}
\setsansfont{Arimo}[Scale=MatchLowercase]
\setmonofont{Cousine}[Scale=MatchLowercase]

\title{The Big Book of Computing}
\author{Jon Eyolfson}

\begin{document}

  \maketitle

  \tableofcontents

  \chapter{Numeral Systems}

  \section{Decimal}

  The number system you're use to is base 10.

  \section{Binary}

  $4 = (100)_2$

  \section{Octal}

  This is the same as before, except now it's base 8.

  \section{Hexadecimal}

  Same as binary and octal, except instead of base 2 you use base 16.

  \begin{tabular}{c c}
    \hline
    Value & Equivalent \\
    \hline
    10 & $(\text{A})_{16}$ \\
    11 & $(\text{B})_{16}$ \\
    12 & $(\text{C})_{16}$ \\
    13 & $(\text{D})_{16}$ \\
    14 & $(\text{E})_{16}$ \\
    15 & $(\text{F})_{16}$ \\
  \end{tabular}

  For instance: $95 = (5\text{F})_{16}$.

  \chapter{Encodings}

  First rule: everything in computing is a number.

  \newpage
  \section{Character}

  \subsection{ASCII}

  {\ttfamily\begin{tabular}{c c}
    \hline
    Value & Meaning \\
    \hline
      0 & \colorbox{gray}{NUL} \\
      1 & \colorbox{gray}{SOH} \\
      2 & \colorbox{gray}{STX} \\
      3 & \colorbox{gray}{ETX} \\
      4 & \colorbox{gray}{EOT} \\
      5 & \colorbox{gray}{ENQ} \\
      6 & \colorbox{gray}{ACK} \\
      7 & \colorbox{gray}{BEL} \\
      8 & \colorbox{gray}{BS} \\
      9 & \colorbox{gray}{HT} \\
     10 & \colorbox{gray}{LF} \\
     11 & \colorbox{gray}{VT} \\
     12 & \colorbox{gray}{FF} \\
     13 & \colorbox{gray}{CR} \\
     14 & \colorbox{gray}{SO} \\
     15 & \colorbox{gray}{SI} \\
     16 & \colorbox{gray}{DLE} \\
     17 & \colorbox{gray}{DC1} \\
     18 & \colorbox{gray}{DC2} \\
     19 & \colorbox{gray}{DC3} \\
     20 & \colorbox{gray}{DC4} \\
     21 & \colorbox{gray}{NAK} \\
     22 & \colorbox{gray}{SYN} \\
     23 & \colorbox{gray}{ETB} \\
     24 & \colorbox{gray}{CAN} \\
     25 & \colorbox{gray}{EM} \\
     26 & \colorbox{gray}{SUB} \\
     27 & \colorbox{gray}{ESC} \\
     28 & \colorbox{gray}{FS} \\
     29 & \colorbox{gray}{GS} \\
     30 & \colorbox{gray}{RS} \\
     31 & \colorbox{gray}{US} \\
  \end{tabular}
  \quad
  \begin{tabular}{c c}
    \hline
    Value & Meaning \\
    \hline
     32 & \colorbox{gray}{SPACE} \\
     33 & ! \\
     34 & " \\
     35 & \# \\
     36 & \$ \\
     37 & \% \\
     38 & \& \\
     39 & ' \\
     40 & ( \\
     41 & ) \\
     42 & * \\
     43 & + \\
     44 & , \\
     45 & - \\
     46 & . \\
     47 & / \\
     48 & 0 \\
     49 & 1 \\
     50 & 2 \\
     51 & 3 \\
     52 & 4 \\
     53 & 5 \\
     54 & 6 \\
     55 & 7 \\
     56 & 8 \\
     57 & 9 \\
     58 & : \\
     59 & ; \\
     60 & < \\
     61 & = \\
     62 & > \\
     63 & ? \\
  \end{tabular}
  \quad
  \begin{tabular}{c c}
    \hline
    Value & Meaning \\
    \hline
     64 & @ \\
     65 & A \\
     66 & B \\
     67 & C \\
     68 & D \\
     69 & E \\
     70 & F \\
     71 & G \\
     72 & H \\
     73 & I \\
     74 & J \\
     75 & K \\
     76 & L \\
     77 & M \\
     78 & N \\
     79 & O \\
     80 & P \\
     81 & Q \\
     82 & R \\
     83 & S \\
     84 & T \\
     85 & U \\
     86 & V \\
     87 & W \\
     88 & X \\
     89 & Y \\
     90 & Z \\
     91 & [ \\
     92 & \textbackslash \\
     93 & ] \\
     94 & \textasciicircum \\
     95 & \_ \\
  \end{tabular}
  \quad
  \begin{tabular}{c c}
    \hline
    Value & Meaning \\
    \hline
     96 & ` \\
     97 & a \\
     98 & b \\
     99 & c \\
    100 & d \\
    101 & e \\
    102 & f \\
    103 & g \\
    104 & h \\
    105 & i \\
    106 & j \\
    107 & k \\
    108 & l \\
    109 & m \\
    110 & n \\
    111 & o \\
    112 & p \\
    113 & q \\
    114 & r \\
    115 & s \\
    116 & t \\
    117 & u \\
    118 & v \\
    119 & w \\
    120 & x \\
    121 & y \\
    122 & z \\
    123 & \{ \\
    124 & | \\
    125 & \} \\
    126 & \textasciitilde \\
    127 & \colorbox{gray}{DEL} \\
  \end{tabular}}

  \newpage
  \section{Data}

  \subsection{Base64}

  The following is a table of the encoded values:

  {\ttfamily\begin{tabular}{c c}
    \hline
    Value & ASCII Character \\
    \hline
     0 & A \\
     1 & B \\
     2 & C \\
     3 & D \\
     4 & E \\
     5 & F \\
     6 & G \\
     7 & H \\
     8 & I \\
     9 & J \\
    10 & K \\
    11 & L \\
    12 & M \\
    13 & N \\
    14 & O \\
    15 & P \\
    16 & Q \\
    17 & R \\
    18 & S \\
    19 & T \\
    20 & U \\
    21 & V \\
    22 & W \\
    23 & X \\
    24 & Y \\
    25 & Z \\
    26 & a \\
    27 & b \\
    28 & c \\
    29 & d \\
    30 & e \\
    31 & f \\
  \end{tabular}
  \quad
  \begin{tabular}{c c}
    \hline
    Value & ASCII Character \\
    \hline
    32 & g \\
    33 & h \\
    34 & i \\
    35 & j \\
    36 & k \\
    37 & l \\
    38 & m \\
    39 & n \\
    40 & o \\
    41 & p \\
    42 & q \\
    43 & r \\
    44 & s \\
    45 & t \\
    46 & u \\
    47 & v \\
    48 & w \\
    49 & x \\
    50 & y \\
    51 & z \\
    52 & 0 \\
    53 & 1 \\
    54 & 2 \\
    55 & 3 \\
    56 & 4 \\
    57 & 5 \\
    58 & 6 \\
    59 & 7 \\
    60 & 8 \\
    61 & 9 \\
    62 & + \\
    63 & / \\
  \end{tabular}}

  \chapter{Programming Languages}

  \section{Assembly}

  \subsection{x86\_64}

  \begin{listing}[H]
    \inputminted[frame=lines]{asm}{code/hello_world.asm}
    \caption{``Hello world'' program written in x86\_64 assembly for Linux}
    \label{lst:hello-world-asm}
  \end{listing}

  Assuming the file is called ``\mintinline{console}{hello_world.asm}'' we need
  to compile to machine code using
  \mintinline{console}{nasm -f elf64 hello_world.asm}. This produces a file
  named ``\mintinline{console}{hello_world.o}'' which is an object file (more
  later). To produce an executable file that will run on the machine use
  \mintinline{console}{ld hello_world.o -o hello_world}. This produces an
  executable named ``\mintinline{console}{hello_world}''. Running the program
  using \mintinline{console}{./hello_world} produces the output
  ``\mintinline{console}{Hello world!}''.

  x86\_64 has 16 general purpose registers: \mintinline{asm}{rax},
  \mintinline{asm}{rbx}, \mintinline{asm}{rcx}, \mintinline{asm}{rdx},
  \mintinline{asm}{rbp}, \mintinline{asm}{rsi}, \mintinline{asm}{rdi},
  \mintinline{asm}{rsp}, \mintinline{asm}{r8}, \mintinline{asm}{r9},
  \mintinline{asm}{r10}, \mintinline{asm}{r11}, \mintinline{asm}{r12},
  \mintinline{asm}{r13}, \mintinline{asm}{r14}, and \mintinline{asm}{r15}. The
  stack pointer is \mintinline{asm}{rsp} and it is always(?) in use.

  Note that the system call number goes in register \mintinline{asm}{rax}. Note
  that (on x86\_64) the system call number for \mintinline{asm}{write} is
  \texttt{1} and for \mintinline{asm}{exit_group} is \texttt{231}. The arguments
  to the system call go in registers according to the following table:

  {\ttfamily\begin{tabular}{c c}
    \hline
    Index & Register \\
    \hline
    0 & \mintinline{asm}{rdi} \\
    1 & \mintinline{asm}{rsi} \\
    2 & \mintinline{asm}{rdx} \\
    3 & \mintinline{asm}{r10} \\
    4 & \mintinline{asm}{r8} \\
    5 & \mintinline{asm}{r9} \\
  \end{tabular}}

  In C, \mintinline{asm}{rcx} is used instead of \mintinline{asm}{r10} for
  the argument at index 3. The kernel destroys registers in
  \mintinline{asm}{rcx} and \mintinline{asm}{r11}. Registers
  \mintinline{asm}{rbp}, \mintinline{asm}{rbx}, \mintinline{asm}{r12},
  \mintinline{asm}{r13}, \mintinline{asm}{r14}, and \mintinline{asm}{r15} belong
  to the calling function and are untouched by the kernel.

  \chapter{Programming}

  \section{Parallelism}

  \subsection{Limitations}

  \subsubsection{Amdahl's Law}

  Let $N$ be the number of parallel executions.

  \noindent Let $S$ be the fraction of serial runtime for a serial execution.

  \noindent Let $P$ be the fraction of parallel runtime for a serial execution.

  \vspace{1em}

  $\text{speedup} = \frac{1}{S + \frac{P}{N}}$

  \subsubsection{Gustafson's Law}

  Let $N$ be the number of parallel executions.

  \noindent Let $n$ be a measure of the problem size.

  \noindent Let $S(n)$ be the fraction of serial runtime for a parallel
  execution.

  \noindent Let $P(n)$ be the fraction of parallel runtime for a parallel
  execution.

  \vspace{1em}

  $\text{speedup} = S(n) + N \cdot P(n)$

  \chapter{Linux Kernel}

  Include memory allocation declarations with
  \mintinline{c}{#include <linux/slab.h>}.

  \chapter{Information Theory}

  \section{Hamming Distance}

  A measure of how many characters are different in two strings of equal length.

  Example: Consider the two ASCII encoded strings ``this is a test'' and ``wokka
  wokka!!!''. Both have the same length of 14 characters. If each character is
  represented as 8 bits we have to many 112 comparsions ($14 \times 8$).

  \texttt{t} is $116 = (74)_{16} = (01110100)_{2}$

  \texttt{w} is $119 = (77)_{16} = (01110111)_{2}$

  This hamming distance between these two characters is 2.

  \texttt{h} is $104 = (68)_{16} = (01101000)_{2}$

  \texttt{o} is $111 = (6\text{F})_{16} = (01101111)_{2}$

  This hamming distance between these two characters is 3, running total of 5.

  \texttt{i} is $105 = (69)_{16} = (01101001)_{2}$

  \texttt{k} is $107 = (6\text{B})_{16} = (01101011)_{2}$

  This hamming distance between these two characters is 1, running total of 6.

  And so on, the hamming distance of these two strings is 37.

\end{document}
