\chapter{Mathematical Basics}

This chapter adapts the basics you've likely learned in elementary school to a
format that's essential for understanding computing. Computers are one of the
dumbest machines in the world: they don't do anything unless you tell them to,
and they only understand two numbers. If you think computers are smart and
magical, you'll quickly learn otherwise.

\section{Numeral Systems}

We have a good understanding how to read and write numbers intuitively since
they're ubiquitous in our day-to-day life. To understand the basics of computing
we'll have to come at it from a slightly different viewpoint. Again, computers
are dumb and we'll have to learn to speak their language. We'll start with an
overview of how we, the smart ones, deal with numbers.

\subsection{Decimal}

The number system we're all use to is decimal, or base 10. If we really break it
down, a number is just a sequence of digits where each digit is one of 10 values
(0 to 9). When we write a number that has more than one digit, we understand the
value of each digit based on where it is in the number. Let's really break it
down.

A number such as 321 can be broken down by looking at the value associated with
every digit. Starting at the rightmost digit, we associate that digit with the
number of 1s in the number. For the next digit, we associate with the number of
10s in the number. Finally, for the leftmost digit in this number, it's
associated with the number of 100s.

So, starting with the rightmost digit, with the association to the number of 1s,
we increase the meaning of the digit to the left by a factor of 10. Of course,
this is where the 10 in base 10 is from. We have 10 digits to choose from and
each digit in a number is associated with a factor of 10. Every digit represents
the number of things of 10 to the power of something (hence base 10). Using
exponents we can write the asssociated number, starting with the rightmost, as
$10^0$, then increase the exponent everytime we move left: $10^1$, $10^2$, and
so on.

This is so normal, you may not have really thought about this in all of the
detail. Get use to this type of thinking, you're going to have to explain
everything to a computer in this level of detail.

We can explicitly chart the values of each digit. In the first row is the value
of each of the digits and below of value is the digit in our number itself. The
chart for $321$ is as follows:

\begin{center}
  \begin{tabular}{r | r | r}
    $10^2$ & $10^1$ & $10^0$ \\
    \hline
         3 &      2 &      1 \\
  \end{tabular}
\end{center}

To get the value of the number, we multiple the value of the digit by the value
of the column it's in. This means we'd do the following calculation:

\begin{center}
  \begin{math}
    \begin{aligned}
&= 10^2 \times 3 + 10^1 \times 2 + 10^0 \times 1 \\
&= 300 + 20 + 1 \\
&= 321 \\
    \end{aligned}
  \end{math}
\end{center}

The notation, if we want to be explicit about the base of the digits, is to
write it in subscript. So, $321$ would be $(321)_{10}$. Since we all understand
decimal however, it's always omitted.

Now, this isn't very useful since we got back the exact same number we wrote.
Again, that's because we already understand decimal numbers. Unfortunately
computers aren't as gifted as us. We'll be using this technique to convert
numbers understood by computers into something we can understand.

\label{sec:numeral-system-binary}
\subsection{Binary}

Computers only understand 2 states, and you can think of them however you want:
on and off, true and false, up and down, or 1 and 0. If we think of them as 1
and 0, we can express numbers, the same as before in base 10, only now in base
2. Instead of writing a series of digits between 0-9, we write a series of bits
(binary digits) between 0-1.

Take the number $(101)_2$ for instance, we do the same thing as we did for
decimal numbers: the rightmost bit has the value of 2 (the base) to the power
of 0, and then we increase the exponent by 1 to get the value of the next bit
to the left. In this case, we'd get the following chart:

\begin{center}
  \begin{tabular}{r | r | r}
    $2^2$ & $2^1$ & $2^0$ \\
    \hline
        1 &      0 &      1 \\
  \end{tabular}
\end{center}

Now, unlikely in base 10, we can't really understand the number directly without
actually doing the multiplication and addition. To get the number to the
decimal form we understand, just crunch the numbers:

\begin{center}
  \begin{math}
    \begin{aligned}
&= 2^2 \times 1 + 2^1 \times 0 + 2^0 \times 1 \\
&= 4 + 0 + 1 \\
&= 5 \\
&= (101)_2 \\
    \end{aligned}
  \end{math}
\end{center}

So, when we see the binary number 101, we can understand this to be the same as
if we wrote 5. To convert a decimal number to binary, we can carry out long
division by the base, and write the reminder as the rightmost digit and the
reminder of the next divison as the digit to the left and so on. To convert 5
back to binary, we'd do the following:

\begin{center}
  \begin{math}
    \begin{aligned}
5/2 &= 2 Reminder 1 \\
2/2 &= 1 Reminder 0 \\
1/2 &= 0 Reminder 1 \\
    \end{aligned}
  \end{math}
\end{center}

Giving $(101)_2$. For practice, consider doing this again for 10:

\begin{center}
  \begin{math}
    \begin{aligned}
10/2 &= 5 Reminder 0 \\
5/2 &= 2 Reminder 1 \\
2/2 &= 1 Reminder 0 \\
1/2 &= 0 Reminder 1 \\
    \end{aligned}
  \end{math}
\end{center}

So, 10 would be $(1010)_2$ in binary. We can use the chart as before to verify
we're correct:

\begin{center}
  \begin{tabular}{r | r | r | r}
    $2^3$ & $2^2$ & $2^1$ & $2^0$ \\
    \hline
        1 &     0 &    1  &     0 \\
  \end{tabular}
\end{center}

Again, just crunch the numbers:

\begin{center}
  \begin{math}
    \begin{aligned}
&= 2^3 \times 1 + 2^2 \times 0 + 2^1 \times 1 + 2^0 \times 0 \\
&= 8 + 2 \\
&= 10 \\
&= (1010)_2 \\
    \end{aligned}
  \end{math}
\end{center}

\subsection{Hexadecimal}

The same lessons apply to hexidecimal numbers, that apply to decimal and binary
numbers, just this time we have 16 options for a single digit. Since we're
using a larger base then we're use to, we have to use more than all the possible
digits we're use to for a single hex digit. So, for the digits with a value
above 9, we use the beginning letters of the alphabet to represent them,
starting at A to represent 10. The full table of their meanings is below:

\begin{tabular}{c c}
  \hline
  Value & Hex Digit \\
  \hline
  10 & $(\text{A})_{16}$ \\
  11 & $(\text{B})_{16}$ \\
  12 & $(\text{C})_{16}$ \\
  13 & $(\text{D})_{16}$ \\
  14 & $(\text{E})_{16}$ \\
  15 & $(\text{F})_{16}$ \\
\end{tabular}

Now, let's do the exact same procedure to find the decimal value of the
hexadecimal number $(5\text{F})_{16}$. Here's our familiar chart:

\begin{center}
  \begin{tabular}{r | r}
    $16^1$ & $16^0$ \\
    \hline
        5  &      F \\
  \end{tabular}
\end{center}

Crunching the numbers again:

\begin{center}
  \begin{math}
    \begin{aligned}
&= 16^1 \times 5 + 16^0 \times 15 \\
&= 80 + 15 \\
&= 95 \\
&= (5\text{F})_{16} \\
    \end{aligned}
  \end{math}
\end{center}

We can similarily take a decimal number and convert it to hexidecimal by doing
the long division technique again.

There's a nice property hexidecimal numbers have when you have to deal with
binary: you can save a lot of space by writing binary numbers as their
hexadecimal equivalents since one hexadecimal digit can be represented as a 4
bit number. For instance, $(5)_{16}$ is $(101)_2$ in binary and
$(\text{F})_{16}$ is $(1111)_2$ in binary. Instead of doing a long conversion to
binary, we can just write their hexadecimal to binary equivalents. This would
give us the binary number $(1011111)_2$ which is equal to $95$.

\section{Terminology}

Understand the following terminology: commutativity, and associativity.
