\chapter{Mathematical Basics}

Understand the following terminology: commutativity, and associativity.

\section{Numeral Systems}

\subsection{Decimal}

The number system you're use to is base 10. A number such as 101 can be broken
down by looking at the value associated with every digit. The numeral system
we're all familiar with is base 10. We start by associating the last digit with
the value 1 ($10^0$) and increase this value by a factor of 10 for each digit
as we work from right to left. We can break down 101 in the following chart:

\begin{center}
  \begin{tabular}{r | r | r}
    $10^2$ & $10^1$ & $10^0$ \\
    \hline
         1 &      0 &      1 \\
  \end{tabular}
\end{center}

\subsection{Binary}

\begin{center}
  \begin{tabular}{r | r | r}
    $2^2$ & $2^1$ & $2^0$ \\
    \hline
        1 &      0 &      1 \\
  \end{tabular}

  \vspace{1em}

  $= 2^2 \times 1 + 2^1 \times 0 + 2^0 \times 1 = 4 \times 1 + 1 \times 1 = 5
   = (101)_2$
\end{center}

\subsection{Octal}

This is the same as before, except now it's base 8.

\subsection{Hexadecimal}

Same as binary and octal, except instead of base 2 you use base 16.

\begin{tabular}{c c}
  \hline
  Value & Equivalent \\
  \hline
  10 & $(\text{A})_{16}$ \\
  11 & $(\text{B})_{16}$ \\
  12 & $(\text{C})_{16}$ \\
  13 & $(\text{D})_{16}$ \\
  14 & $(\text{E})_{16}$ \\
  15 & $(\text{F})_{16}$ \\
\end{tabular}

For instance: $95 = (5\text{F})_{16}$.
