\chapter{Static Analysis}

\section{Terminology}

\begin{description}
  \item[Soundness.] Never report a definite results when the opposite is true.
  \item[Precision.] Reporting a definite result as often as possible.
  \item[Conservative.] Just report ``maybe'' when in doubt.
  \item[Poset.] A partially ordered set.
  \item[Lattice.] Poset with a least upper bound (join operation) and greatest
                  lower bound (meet operation).
  \item[Dataflow Anaylsis.] Iterate over the CFG with flow equations until you
                            reach a fixed point.
\end{description}

Suppose $R$ is a relation on set $A$.

$R$ is reflexive if $\forall x \in A (x R x)$

$R$ is reflexive iff $i_A \subseteq R$, where $i_A$ is the identity  relation
on A.

$R$ is antisymmetric if $\forall x \in A \forall y \in A ((x R y \land y R x)
\rightarrow x = y)$.

$R$ is transitive if $\forall x \in A \forall y \in A \forall z \in A
((x R y \land y R z) \rightarrow x R z)$.

$R$ is transitive iff $R \circ R \subseteq R$.

If $R$ is reflexive, antisymmetric and transitive it is a partial order.

Suppose $R$ is a partial order on a set $A$, $B \subseteq A$, and $b \in B$.
Then $b$ is called an $R$-smallest element of $B$ (or smallest element of $B$)
if $\forall x \in B (b R x)$. It is called  an $R$-minimal element (or just a
minimal element) if $\lnot \exists x \in B (x R b \land x \neq b)$.

Suppose $R$ is a partial order on set $A$, $B \subseteq A$, and $a \in A$ (note
we consider elements in $A$ instead of $B$ as before). Then $a$ is called a
lower bound for $B$ if $\forall x \in B (a R x)$. Similarly, it is an upper
bound for $B$ if $\forall x \in B (x R a)$.

Suppose $R$ is a partial order on set $A$, $B \subseteq A$. Let $U$ be the set
of all upper bounds for $B$, and let $L$ be the set of all lower bounds. If $U$
has a smallest element, then this smallest element is called the least upper
bound of $B$. If $L$ has a largest element, then this largest element is called
the greatest lower bound of $B$.

These are all part of set theory in mathematics. A lot of proofs and theory in
static analysis relies on a solid understanding over the underlaying
mathematics.

\section{Control Flow Graph (CFG)}

A CFG shows how program execution occurs. It has an entry edge where execution
starts. It can have multiple exit edges. Typically in programming languages this
exit is a return statement.

Consider below, and 1 is the entry node. This diagram has two exit nodes, 2 and
5. There is a loop in nodes 3 and 4.

\begin{tikzpicture}[->,>=stealth']
  \node (1) {1};
  \node (2) [below of=1, xshift=-5mm] {2};
  \node (3) [right of=2] {3};
  \node (4) [below of=3, xshift=-5mm] {4};
  \node (5) [right of=4] {5};

  \path (1) edge (2)
        (1) edge (3)
        (3) edge (4)
        (4) edge [bend right] (3)
        (3) edge (5);
\end{tikzpicture}

This easier to see what possible paths the program may take. We may also want to
know what variables are valid in which blocks.

\section{Dominator Trees}

\begin{tikzpicture}
  \node {1}
    child { node {2}
    }
    child { node {3}
      child { node {4} 
        child { node {5} }
        child { node {6} }
        child { node {7} }
      }
    };
\end{tikzpicture}

\section{IFDS}

Interprocedural, Finite, Distributive, Subset (IFDS).

\section{Alias Analysis}
