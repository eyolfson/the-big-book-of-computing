\chapter{Wayland}

Wayland is a protocol for graphics and input intended for applications.

\textit{Background knowledge: UNIX domain stream socket.}

\begin{tikzpicture}[->,>=stealth',shorten >=0.5mm,grow=right,
                    level distance=2.68cm,sibling distance=1cm]
  \tikzstyle{core} = [draw, rectangle, fill=blue!50]
  \tikzstyle{global} = [draw, rectangle, fill=green!50]
  \node [core] {wl\_display}
    child {
      node [core] {wl\_registry}
        child {
          node [global] {wl\_compositor}
            child {
              node {wl\_surface}
            }
            child {
              node {wl\_region}
            }
        }
        child {
          node [global] {wl\_subcompositor}
        }
        child {
          node [global] {wl\_scaler}
        }
        child {
          node [global] {wl\_data\_device\_manager}
        }
        child {
          node [global] {wl\_text\_input\_manager}
        }
        child {
          node [global] {wl\_shm}
            child {
              node {wl\_shm\_pool}
            }
        }
        child {
          node [global] {wl\_drm}
        }
        child {
          node [global] {wl\_seat}
        }
        child {
          node [global] {wl\_input\_method}
        }
        child {
          node [global] {wl\_output}
        }
        child {
          node [global] {wl\_input\_panel}
        }
        child {
          node [global] {wl\_shell}
            child {
              node {wl\_shell\_surface}
            }
        }
    }
  ;
\end{tikzpicture}

Requests and events.

\section{wl\_output}

\subsection{Events}

\paragraph{geometry}

\begin{enumerate}
  \item x
  \item y
  \item physical\_width (unit: mm)
  \item physical\_height (unit: mm)
  \item subpixel
  \item make
  \item model
  \item transform
\end{enumerate}

