\chapter{Teensy 3.2}

\section{Hardware}

System on chip (SoC) is MK20DX256VLH7.

\begin{description}
  \item[CPU.] ARM Cortex M4
  \item[Storage.] 256 K Flash
  \item[RAM.] 64 K
  \item[ROM.] 2 K EEPROM
\end{description}

\begin{description}
  \item[LP38691.] 500mA Low Dropout CMOS Linear Regulators Stable with Ceramic
                  Output Capacitors
  \item[MKL02Z32VFG4.] ARM Cortex-M0+ Kinetis KL02 Microcontroller IC 32-Bit
                       48MHz 32KB (32K x 8) FLASH 16-QFN (3x3)
  \item[MK20DX256VLH7.] ARM Cortex-M4 Kinetis K20 Microcontroller IC 32-Bit
                        72MHz 256KB (256K x 8) FLASH 64-LQFP (10x10)
\end{description}

The is a secondary CPU containing the \textit{HalfKay bootloader}.
A special USB packet or pressing the on-board button activates this CPU.
It uploads the bootloader code to the main RAM and tells the main CPU to
execute it(?).
The bootloader clears flash and writes the flash with new code, afterwards it
reboots itself.

PTA18 and PTA19 are connected to a 16 MHz clock.

CPU is at 48 MHz or 48,000,000 times a second.

\texttt{SYST\_RVR} is 47,999 or \hexadecimal{BB7F}.
Since it counts down to 0, the SysTick interrupt occurs at 1,000 kHz.
In other words it occurs every ms (millisecond).

\texttt{SYST\_CSR} is \hexadecimal{7}.
Core clock is used for systick.
Generates a SysTick interrupt when 0.
Counter operates in multi-shot manner.
