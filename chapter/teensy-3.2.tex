\chapter{Teensy 3.2}

\section{Hardware}

System on chip (SoC) is MK20DX256VLH7.

\begin{description}
  \item[CPU.] ARM Cortex M4
  \item[Storage.] 256 K Flash
  \item[RAM.] 64 K
  \item[ROM.] 2 K EEPROM
\end{description}

Components:

\begin{description}
  \item[LP38691.] (1) 500 mA Low Dropout CMOS Linear Regulators Stable with
                  Ceramic Output Capacitors
  \item[MKL02Z32VFG4.] (1) ARM Cortex-M0+ Kinetis KL02 Microcontroller IC 32-Bit
                       48MHz 32KB (32K x 8) FLASH 16-QFN (3x3)
  \item[MK20DX256VLH7.] ARM Cortex-M4 Kinetis K20 Microcontroller IC 32-Bit
                        72MHz 256KB (256K x 8) FLASH 64-LQFP
                        (10x10)
\end{description}

The is a secondary CPU containing the \textit{HalfKay bootloader}.
A special USB packet or pressing the on-board button activates this CPU.
It uploads the bootloader code to the main RAM and tells the main CPU to
execute it(?).
The bootloader clears flash and writes the flash with new code, afterwards it
reboots itself.

PTA18 and PTA19 are connected to a 16 MHz clock.

CPU is at 48 MHz or 48,000,000 times a second.

\texttt{SYST\_RVR} is 47,999 or \hexadecimal{BB7F}.
Since it counts down to 0, the SysTick interrupt occurs at 1,000 kHz.
In other words it occurs every ms (millisecond).

\texttt{SYST\_CSR} is \hexadecimal{7}.
Core clock is used for systick.
Generates a SysTick interrupt when 0.
Counter operates in multi-shot manner.

\newpage
\section{Blinking LED}

The hex file to transfer to the Teensy is actually 14324 bytes.
Let's see if we can account for all 14324 bytes.
All instructions are at least 2 bytes, giving a maximum of 7162 instructions.

The address \hexadecimal{00 00 00 00} is the vector table. This table contains
addresses for each entry. Only entry 0, 1, and 15 are important for now.

Minimum alignment is 128 bytes (32 addresses). The first entry is the stack
pointer, all others are destinations of a branch and must have bit 0 set to 1.

[\hexadecimal{00 00 31 F0}, \hexadecimal{00 00 37 F4}) are part of the data
section. This accounts for 1540 bytes. It's copied to
[\hexadecimal{1F FF 84 40}, \hexadecimal{1F FF 8A 44}).

The zero'ed memory section does not take up any space in the executable itself
but does exist in RAM. The zero'ed section is the addresses in the range
[\hexadecimal{1F FF 8A 44}, \hexadecimal{1F FF 8E 2C}).
This takes up 1000 bytes.

Current accounting for \textbf{1134 bytes} in functions.

\subsection{Execution}

The following is the program flow:

\indent \texttt{ResetHandler (mk20dx128.c)}

\hspace{2mm} \texttt{startup\_early\_hook (mk20dx128.c)}

\hspace{2mm} \texttt{\_init\_Teensyduino\_internal\_ (pins\_teensy.c)}

\hspace{4mm} \texttt{analog\_init (analog.c)}

\hspace{4mm} \texttt{delay(4) (pins\_teensy.c)}

\hspace{4mm} \texttt{usb\_init (usb\_dev.c)}

\hspace{6mm} \texttt{usb\_init\_serialnumber (usb\_desc.c)}

\hspace{8mm} \texttt{ultoa (nonstd.c)}

\hspace{2mm} \texttt{\_\_libc\_init\_array (???)}

\hspace{4mm} \texttt{\_init? (???)}

\hspace{4mm} \texttt{\_\_init\_array\_start[0] (???)}

\hspace{6mm} \texttt{unknown1 (???)}

\hspace{4mm} \texttt{\_\_init\_array\_start[1] (???)}

\hspace{2mm} \texttt{startup\_late\_hook (mk20dx128.c)}

\hspace{2mm} \texttt{main (main.cpp)}

\hspace{4mm} \texttt{setup (???)}

\hspace{6mm} \texttt{pinMode (pins\_teensy.c)}

\subsection{Functions}

\paragraph{\texttt{ResetHandler (mk20dx128.c)}} Reset interrupt handler.

\hexadecimal{00 00 01 BC} is load watchdog address.

\vspace{1em}

\hexadecimal{00 00 02 0C} is data copy loop setup start.

\hexadecimal{00 00 02 14} is the initial branch into the conditional check.

\hexadecimal{00 00 02 16} is the start of the loop body.

\hexadecimal{00 00 02 1A} is the end of the loop body.

\hexadecimal{00 00 02 1C} is the start of the conditional check.

\hexadecimal{00 00 02 20} conditional branch back into loop.

\vspace{1em}

\hexadecimal{00 00 02 22} is data copy loop setup start.

\hexadecimal{00 00 02 30} conditional branch back into loop.

\vspace{1em}

\hexadecimal{00 00 02 44} conditional branch back into loop (int. priority).

\vspace{1em}

\hexadecimal{00 00 02 D8} is \texttt{\_\_enable\_irq}.

\hexadecimal{00 00 02 DA} call to \texttt{\_init\_Teensyduino\_internal\_}

\hexadecimal{00 00 02 DE} return from \texttt{\_init\_Teensyduino\_internal\_} (actually \texttt{usb\_init})

\hexadecimal{00 00 02 EC} call to \texttt{\_\_libc\_init\_array}

\hexadecimal{00 00 02 F0} call to \texttt{startup\_late\_hook}

\hexadecimal{00 00 02 F4} call to \texttt{main}

\vspace{1em}

\hexadecimal{00 00 03 10} is \texttt{RTC\_SR} address (4 bytes).

\hexadecimal{00 00 03 40} is \texttt{MCG\_S} address (4 bytes).

\hexadecimal{00 00 03 44} is \texttt{MCG\_C5} address (4 bytes).

\hexadecimal{00 00 03 54} is \texttt{SIM\_CLKDIV2} address (4 bytes).

\hexadecimal{00 00 03 58} is \texttt{SIM\_SOPT2} constant value (4 bytes).

\hexadecimal{00 00 03 5C} is \texttt{SYST\_RVR} address (4 bytes).

\paragraph{\texttt{unknown1 (???)}} Loads data end (or zero'ed memory start) and
is a noop.

\hexadecimal{00 00 04 38} is entry point (push \texttt{R3}, \texttt{R14}).

\hexadecimal{00 00 04 56} is exit point (pop \texttt{R3}, \texttt{R15}).

\textbf{Total: 32 bytes}

\paragraph{\texttt{\_\_init\_array\_start[0] (???)}} Array initialization.

\hexadecimal{00 00 04 98} is entry point (push \texttt{R3}, \texttt{R14}).

\hexadecimal{00 00 04 D4} is call to \texttt{unknown1}.

\textbf{Total: 64 bytes}

\paragraph{\texttt{setup (???)}} Sets the pin's operation.

\hexadecimal{00 00 04 D8} is entry point.

\hexadecimal{00 00 04 DC} is call to \texttt{pinMode}.

\paragraph{\texttt{??? (???)}} Enables Port A-E IRQs.

\hexadecimal{00 00 05 08} is entry point.

\hexadecimal{00 00 05 28} is exit point.

\hexadecimal{00 00 05 2A} is empty padding (2 bytes) [?].

\hexadecimal{00 00 05 2C} is \texttt{NVIC\_ISER2} address (4 bytes).

\textbf{Total: 40 bytes}

\paragraph{\texttt{pinMode (pins\_teensy.c)}} Sets the pin's operation.

\hexadecimal{00 00 07 D8} is entry point.

\hexadecimal{00 00 08 0A} is exit point.

\hexadecimal{00 00 08 0C} is address of \texttt{?} (\hexadecimal{00 00 2E EC}).

\textbf{Total: 56 bytes}

\paragraph{\texttt{micros (pins\_teensy.c)}} Returns current time in
microseconds.

\hexadecimal{00 00 08 10} is \texttt{\_\_disable\_irq}.

\hexadecimal{00 00 08 1E} is \texttt{\_\_enable\_irq}.

\hexadecimal{00 00 08 3E} is \texttt{return ...}.

\vspace{1em}

\hexadecimal{00 00 08 40} is \texttt{SYST\_CVR} address (4 bytes).

\hexadecimal{00 00 08 44} is \texttt{ICSR} address (4 bytes).

\hexadecimal{00 00 08 48} is \texttt{systick\_millis\_count} address (4 bytes).

\textbf{Total: 60 bytes}

\paragraph{\texttt{delay (pins\_teensy.c)}} Delays the processor at least
specified number of milliseconds.

\hexadecimal{00 00 08 4C} is entry point (push \texttt{R4}, \texttt{R5},
\texttt{R6}, \texttt{R14}).

\hexadecimal{00 00 08 50} is call to \texttt{micros()} for \texttt{start}.

\hexadecimal{00 00 08 5C} is call to \texttt{micros()} in \texttt{while}.

\hexadecimal{00 00 08 6E} is \texttt{yield()}.

\hexadecimal{00 00 08 68} is conditional branch to \texttt{return}.

\hexadecimal{00 00 08 74} is \texttt{return} (with \texttt{POP}).

\hexadecimal{00 00 08 76} is empty padding (2 bytes) [?].

\textbf{Total: 44 bytes}

\paragraph{\texttt{\_init\_Teensyduino\_internal\_ (pins\_teensy.c)}}
Init.

\hexadecimal{00 00 08 78} is entry (push \texttt{R4}, \texttt{R14}).

\hexadecimal{00 00 08 7A} is branch to enable port A-E IRQs.

\hexadecimal{00 00 08 CA} call to \texttt{delay} (4 bytes)

\hexadecimal{00 00 08 CE} return from \texttt{delay} (4 bytes) (pop \texttt{R4},
\texttt{R5}, \texttt{R6}, \texttt{R14}).

\hexadecimal{00 00 08 D2} is branch to \texttt{usb\_init}.

\vspace{1em}

\hexadecimal{00 00 08 D8} is \texttt{FTM0\_CNT} address (4 bytes).

\hexadecimal{00 00 08 DC} is \texttt{FTM0\_C0SC} address (4 bytes).

\hexadecimal{00 00 08 E0} is \texttt{FTM0\_SC} address (4 bytes).

\hexadecimal{00 00 08 E4} is \texttt{FTM1\_CNT} address (4 bytes).

\hexadecimal{00 00 08 E8} is \texttt{FTM2\_CNT} address (4 bytes).

\hexadecimal{00 00 08 EC} is \texttt{FTM2\_MOD} address (4 bytes).

\hexadecimal{00 00 08 F0} is \texttt{FTM2\_C0SC} address (4 bytes).

\hexadecimal{00 00 08 F4} is \texttt{FTM2\_SC} address (4 bytes).

\textbf{Total: 128 bytes}

\paragraph{\texttt{yield (???)}} Unknown.

\hexadecimal{00 00 10 00} is an immediate return.

\textbf{Total: 2 bytes}

\paragraph{\texttt{startup\_early\_hook (mk20dx128.c)}} Default, allows updates
to the watchdog.

\hexadecimal{00 00 13 88} is entry point.

\hexadecimal{00 00 13 8E} is \texttt{return}.

\hexadecimal{00 00 13 90} is \texttt{WDOG\_STCTRLH} address.

\paragraph{\texttt{startup\_late\_hook (mk20dx128.c)}} Default is no-op.

\hexadecimal{00 00 13 94} is entry and exit point.

\textbf{Total: 2 bytes}

\paragraph{\texttt{usb\_init\_serialnumber (usb\_desc.c)}} Set serial number.

\hexadecimal{00 00 13 E4} is entry point.
(push \texttt{R0}, \texttt{R1}, \texttt{R2}, \texttt{R3},
\texttt{R4}, \texttt{R14})

\hexadecimal{00 00 14 0E} is branch to \texttt{ultoa}.

\hexadecimal{00 00 14 12} is return from \texttt{ultoa}.

\hexadecimal{00 00 14 30} is move stack pointer up 5 values.

\hexadecimal{00 00 14 32} is return (pop \texttt{R15}).

\vspace{1em}

\hexadecimal{00 00 14 34} is \texttt{FTFL\_FSTAT} address (4 bytes).

\hexadecimal{00 00 14 38} is \texttt{FTFL\_FCCOB0} address (4 bytes).

\hexadecimal{00 00 14 3C} is \texttt{FTFL\_FCCOB7} address (4 bytes).

\hexadecimal{00 00 14 40} is {\tiny
\texttt{usb\_string\_serial\_number\_default}} address (4 bytes).

\textbf{Total: 96 bytes}

\paragraph{\texttt{usb\_init (usb\_dev.c)}} \texttt{NUM\_ENDPOINTS} is 6.
Zeros \texttt{table} from [\hexadecimal{1F FF 80 00},
\hexadecimal{1F FF 80 C8}) which is 200 bytes.

\hexadecimal{00 00 1C 0C} is entry point (push \texttt{R3} and \texttt{R14}).

\hexadecimal{00 00 1C 0E} is branch to \texttt{usb\_init\_serialnumber}.

\hexadecimal{00 00 1C 12} is return from \texttt{usb\_init\_serialnumber}.

\hexadecimal{00 00 1C 8E} is exit point (pop \texttt{R3} and \texttt{R15}).

\hexadecimal{00 00 1C 90} is address of \texttt{table}.

\hexadecimal{00 00 1C 94} is address of \texttt{SIM\_SCGC4}.

\hexadecimal{00 00 1C 98} is address of \texttt{USB0\_USBTRC0}.

\hexadecimal{00 00 1C 9C} is address of \texttt{USB0\_BDTPAGE1}.

\hexadecimal{00 00 1C A0} is address of \texttt{NVIC\_IPR18} (Number 73).

\hexadecimal{00 00 1C A4} is address of \texttt{USB0\_CONTROL}.

\textbf{Total: 156 bytes}

\paragraph{\texttt{ultoa (nonstd.c)}} Convert integer to string (with radix).

\hexadecimal{00 00 1D 28} is entry point (push \texttt{R4} and \texttt{R14}).

\hexadecimal{00 00 1D 62} is exit (pop \texttt{R4} and \texttt{R15}) (2 bytes).

\textbf{Total: 60 bytes}

\paragraph{\texttt{analog\_init (analog.c)}} Unknown

\hexadecimal{00 00 20 9C} is entry point.

\hexadecimal{00 00 21 26} is exit point.

\hexadecimal{00 00 21 28} is \texttt{VREF\_TRM} address (4 bytes).

\hexadecimal{00 00 21 2C} is \texttt{analog\_config\_bits} address (4 bytes).

\hexadecimal{00 00 21 30} is \texttt{ADC0\_CFG1} address (4 bytes).

\hexadecimal{00 00 21 34} is \texttt{ADC0\_CFG2} address (4 bytes).

\hexadecimal{00 00 21 38} is \texttt{ADC1\_CFG1} address (4 bytes).

\hexadecimal{00 00 21 40} is \texttt{analog\_reference\_internal} address
(4 bytes).

\hexadecimal{00 00 21 44} is \texttt{ADC0\_SC2} address (4 bytes).

\hexadecimal{00 00 21 48} is \texttt{ADC1\_SC2} address (4 bytes).

\hexadecimal{00 00 21 4C} is \texttt{analog\_num\_average} address (4 bytes).

\hexadecimal{00 00 21 50} is \texttt{ADC0\_SC3} address (4 bytes).

\hexadecimal{00 00 21 54} is \texttt{ADC1\_SC3} address (4 bytes).

\hexadecimal{00 00 21 58} is \texttt{calibrating} address (4 bytes).

\textbf{Total: 192 bytes}

\paragraph{\texttt{\_\_init\_array\_start[1] (???)}} Array initialization.

\hexadecimal{00 00 22 60} is entry point.

\hexadecimal{00 00 22 72} is exit point.

\hexadecimal{00 00 22 74} is \texttt{unknownvar1} address (4 bytes).

\hexadecimal{00 00 22 78} is constant value address (4 bytes).

\textbf{Total: 28 bytes}

\paragraph{\texttt{\_\_init\_array\_start[2] (???)}} Array initialization.

\hexadecimal{00 00 23 58} is entry point.

\hexadecimal{00 00 23 62} is branch to \texttt{unknown2}.

\hexadecimal{00 00 23 68} is \texttt{unknownvar2} address (4 bytes).

\hexadecimal{00 00 23 6C} is constant value address (4 bytes).

\hexadecimal{00 00 23 70} is \texttt{unknownvar3} address (4 bytes).

\paragraph{\texttt{\_\_init\_array\_start[3] (???)}} Array initialization.

\hexadecimal{00 00 24 48} is entry point.

\hexadecimal{00 00 24 5A} is exit point.

\textbf{Total: 20 bytes}

\paragraph{\texttt{\_\_init\_array\_start[4] (???)}} Array initialization.

\hexadecimal{00 00 24 A8} is entry point.

\hexadecimal{00 00 24 B8} is exit point.

\textbf{Total: 18 bytes}

\paragraph{\texttt{\_\_init\_array\_start[5] (???)}} Array initialization.

\hexadecimal{00 00 24 C4} is entry point.

\hexadecimal{00 00 24 D6} is exit point.

\textbf{Total: 20 bytes}

\paragraph{\texttt{main (main.cpp)}} The main function.

\hexadecimal{00 00 24 E0} is entry point (push \texttt{R3}, \texttt{R14}).

\hexadecimal{00 00 24 E2} is call to \texttt{setup}.

\paragraph{\texttt{\_\_init\_array\_start[6] (???)}} Array initialization.

\hexadecimal{00 00 25 58} is entry point.

\hexadecimal{00 00 25 6A} is exit point.

\textbf{Total: 20 bytes}

\paragraph{\texttt{unknown2 (???)}} No idea.

\hexadecimal{00 00 25 74} is entry point.

\hexadecimal{00 00 25 90} is call to \texttt{unknown3}.

\paragraph{\texttt{\_\_libc\_init\_array (???)}} Provided by C. It seems the
function pointer array,  \texttt{\_\_init\_array\_start}, starts at
\hexadecimal{31 D4} and ends at \hexadecimal{31 F0} (which is the beginning of
data). This is \hexadecimal{1C} bytes long, or 7 function pointers.

\hexadecimal{00 00 25 94} is entry point (push \texttt{R4}, \texttt{R5},
\texttt{R6}, \texttt{R14}).

\hexadecimal{00 00 25 CE} is call to \texttt{\_init?}.

\hexadecimal{00 00 25 D2} is return from \texttt{\_init?}.

\hexadecimal{00 00 25 E0} is call to \texttt{\_\_init\_array\_start[i]}.

\hexadecimal{00 00 25 E2} is return from \texttt{\_\_init\_array\_start[i]}.

\hexadecimal{00 00 25 E6} is exit point (pop \texttt{R4}, \texttt{R5},
\texttt{R6}, \texttt{R15}).

\textbf{Total: 84 bytes}

\paragraph{\texttt{unknown3 (???)}} No idea.

\hexadecimal{00 00 2B D4} is entry point (push \texttt{R4}, \texttt{R5},
\texttt{R6}, \texttt{R7}, \texttt{R8}, \texttt{R14}).

\hexadecimal{00 00 2C 78} is loop branch back to \hexadecimal{00 00 2B EC}.

\hexadecimal{00 00 31 60} is address of \texttt{unknownvar4}.

\paragraph{\texttt{\_init? (???)}} Best guess.

\hexadecimal{00 00 31 C8} is entry point (push \texttt{R3}, \texttt{R4},
\texttt{R5}, \texttt{R6}, \texttt{R7}, \texttt{R14}).

\hexadecimal{00 00 31 CC} is pop \texttt{R3}, \texttt{R4}, \texttt{R5},
\texttt{R6}, \texttt{R7}.

\hexadecimal{00 00 31 CE} is pop \texttt{R3} (old \texttt{R14}).

\hexadecimal{00 00 31 D0} is move \texttt{R3} to \texttt{R14}.

\hexadecimal{00 00 31 D2} is exit.

\subsection{Variables}

Below are all the variables:

\vspace{1em}

\hexadecimal{1F FF 80 00} is USB descriptor table \texttt{table} (224 bytes).

\hexadecimal{1F FF 84 40} is \texttt{unknownvar3} (? bytes).

\hexadecimal{1F FF 85 22} is {\tiny
\texttt{usb\_string\_serial\_number\_default}} struct (12 bytes).

\hexadecimal{1F FF 85 44} is \texttt{unknownvar4} (? bytes).

\hexadecimal{1F FF 85 3C} is \texttt{analog\_config\_bits} value (1 byte).

\hexadecimal{1F FF 85 3D} is \texttt{analog\_num\_average} (1 byte).

\hexadecimal{1F FF 85 8C} is \texttt{???} (= \hexadecimal{1F FF 85 90}) (4 bytes).

\hexadecimal{1F FF 8A E8} is \texttt{systick\_millis\_count} (4 bytes).

\hexadecimal{1F FF 8D 6A} is \texttt{calibrating} (1 byte).

\hexadecimal{1F FF 8D 6B} is \texttt{analog\_reference\_internal} (1 byte).

\hexadecimal{1F FF 8D 74} is \texttt{unknowvar1} (4 byte) (= \hexadecimal{00 00 30 90}).

\hexadecimal{1F FF 8D 78} is \texttt{unknowvar1} (1 byte) (= \hexadecimal{00}).

\hexadecimal{1F FF 8D 7C} is \texttt{unknowvar1} (4 bytes) (= \hexadecimal{00 00 03 E8}).

\hexadecimal{1F FF 8D 80} is \texttt{unknowvar1} (1 byte) (= \hexadecimal{00}).

\hexadecimal{1F FF 8D 9C} is \texttt{unknownvar2} (? bytes).

\hexadecimal{?? ?? ?? ??} is \texttt{analog\_right\_shift} (1 byte).
