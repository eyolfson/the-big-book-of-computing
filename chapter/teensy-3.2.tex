\chapter{Teensy 3.2}

\section{Hardware}

System on chip (SoC) is MK20DX256VLH7.

\begin{description}
  \item[CPU.] ARM Cortex M4
  \item[Storage.] 256 K Flash
  \item[RAM.] 64 K
  \item[ROM.] 2 K EEPROM
\end{description}

Components:

\begin{description}
  \item[LP38691.] (1) 500 mA Low Dropout CMOS Linear Regulators Stable with
                  Ceramic Output Capacitors
  \item[MKL02Z32VFG4.] (1) ARM Cortex-M0+ Kinetis KL02 Microcontroller IC 32-Bit
                       48MHz 32KB (32K x 8) FLASH 16-QFN (3x3)
  \item[MK20DX256VLH7.] ARM Cortex-M4 Kinetis K20 Microcontroller IC 32-Bit
                        72MHz 256KB (256K x 8) FLASH 64-LQFP
                        (10x10)
\end{description}

The is a secondary CPU containing the \textit{HalfKay bootloader}.
A special USB packet or pressing the on-board button activates this CPU.
It uploads the bootloader code to the main RAM and tells the main CPU to
execute it(?).
The bootloader clears flash and writes the flash with new code, afterwards it
reboots itself.

PTA18 and PTA19 are connected to a 16 MHz clock.

CPU is at 48 MHz or 48,000,000 times a second.

\texttt{SYST\_RVR} is 47,999 or \hexadecimal{BB7F}.
Since it counts down to 0, the SysTick interrupt occurs at 1,000 kHz.
In other words it occurs every ms (millisecond).

\texttt{SYST\_CSR} is \hexadecimal{7}.
Core clock is used for systick.
Generates a SysTick interrupt when 0.
Counter operates in multi-shot manner.

\newpage
\section{Blinking LED}

The hex file to transfer to the Teensy is actually 14324 bytes.
Let's see if we can account for all 14324 bytes.
All instructions are at least 2 bytes, giving a maximum of 7162 instructions.

The address \hexadecimal{00 00 00 00} is the vector table. This table contains
addresses for each entry. Only entry 0, 1, and 15 are important for now.

Minimum alignment is 128 bytes (32 addresses). The first entry is the stack
pointer, all others are destinations of a branch and must have bit 0 set to 1.

[\hexadecimal{00 00 31 F0}, \hexadecimal{00 00 37 F4}) are part of the data
section. This accounts for 1540 bytes. It's copied to
[\hexadecimal{1F FF 84 40}, \hexadecimal{1F FF 8A 44}).

The zero'ed memory section does not take up any space in the executable itself
but does exist in RAM. The zero'ed section is the addresses in the range
[\hexadecimal{1F FF 8A 44}, \hexadecimal{1F FF 8E 2C}).
This takes up 1000 bytes.

\paragraph{\texttt{ResetHandler (mk20dx128.c)}} Reset interrupt handler.

\hexadecimal{00 00 01 BC} is load watchdog address.

\vspace{1em}

\hexadecimal{00 00 02 0C} is data copy loop setup start.

\hexadecimal{00 00 02 14} is the initial branch into the conditional check.

\hexadecimal{00 00 02 16} is the start of the loop body.

\hexadecimal{00 00 02 1A} is the end of the loop body.

\hexadecimal{00 00 02 1C} is the start of the conditional check.

\hexadecimal{00 00 02 20} conditional branch back into loop.

\vspace{1em}

\hexadecimal{00 00 02 22} is data copy loop setup start.

\hexadecimal{00 00 02 30} conditional branch back into loop.

\vspace{1em}

\hexadecimal{00 00 02 44} conditional branch back into loop (int. priority).

\vspace{1em}

\hexadecimal{00 00 02 D8} is \texttt{\_\_enable\_irq}.

\paragraph{\texttt{startup\_early\_hook (mk20dx128.c)}} Default, allows updates
to the watchdog.

\hexadecimal{00 00 13 88} is entry point.

\hexadecimal{00 00 13 8E} is \texttt{return}.

\paragraph{\texttt{delay (pins\_teensy.c)}} Delays the processor at least
specified number of milliseconds.

\hexadecimal{00 00 08 5C} is call to \texttt{micros()} in \texttt{while}.

\hexadecimal{00 00 08 6E} is \texttt{yield()}.

\hexadecimal{00 00 08 68} is conditional branch to \texttt{return}.

\hexadecimal{00 00 08 74} is \texttt{return} (with \texttt{POP}).

\paragraph{\texttt{micros (pins\_teensy.c)}} Returns current time in
microseconds.

\hexadecimal{00 00 08 10} is \texttt{\_\_disable\_irq}.

\hexadecimal{00 00 08 1E} is \texttt{\_\_enable\_irq}.

\hexadecimal{00 00 08 3E} is \texttt{return ...}.

\paragraph{\texttt{??? (???)}} Unknown

\hexadecimal{00 00 08 CE} is \texttt{return ...} (with \texttt{POP}).

\paragraph{\texttt{??? (???)}} Unknown

\hexadecimal{00 00 08 D2} is branch to ???.

\paragraph{\texttt{yield (???)}} Unknown

\hexadecimal{00 00 10 00} is an immediate return.

\subsection{Variables}

Below are all the variables:

\hexadecimal{00 00 03 40} is \texttt{MCG\_S} address (4 bytes).

\hexadecimal{00 00 03 44} is \texttt{MCG\_C5} address (4 bytes).

\hexadecimal{00 00 03 54} is \texttt{SIM\_CLKDIV2} address (4 bytes).

\hexadecimal{00 00 03 58} is \texttt{SIM\_SOPT2} constant value (4 bytes).

\hexadecimal{00 00 03 5C} is \texttt{SYST\_RVR} address (4 bytes).

\hexadecimal{1F FF 8A E8} is \texttt{systick\_millis\_count} (4 bytes).
