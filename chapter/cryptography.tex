\chapter{Cryptography}

\section{Information Theory}

\subsection{Hamming Distance}

A measure of how many characters are different in two strings of equal length.

Example: Consider the two ASCII encoded strings ``this is a test'' and ``wokka
wokka!!!''. Both have the same length of 14 characters. If each character is
represented as 8 bits we have to many 112 comparsions ($14 \times 8$).

\texttt{t} is $116 = (74)_{16} = (01110100)_{2}$

\texttt{w} is $119 = (77)_{16} = (01110111)_{2}$

This hamming distance between these two characters is 2.

\texttt{h} is $104 = (68)_{16} = (01101000)_{2}$

\texttt{o} is $111 = (6\text{F})_{16} = (01101111)_{2}$

This hamming distance between these two characters is 3, running total of 5.

\texttt{i} is $105 = (69)_{16} = (01101001)_{2}$

\texttt{k} is $107 = (6\text{B})_{16} = (01101011)_{2}$

This hamming distance between these two characters is 1, running total of 6.

And so on, the hamming distance of these two strings is 37.

\section{Block Ciphers}

\subsection{Modes of Operation}

\subsubsection{Electronic Codebook (ECB)}

\subsubsection{Cipher Block Chaining (CBC)}

\subsubsection{Counter (CTR)}

\section{Hashing}

\subsection{AES}
