\chapter{Arithmetic}

Example of floor:

$\lfloor 0.8 \rfloor = 0$

Example of ceiling:

$\lceil 0.2 \rceil = 1$

Truncation:

$\text{trunc}(x, n)$

\section{Modulo}

Dividend / Divisor = (Quotient, Remainder)

$a / b = (q, r)$

where

$a = bq + r$

there's a unique answer for $a, b$ when $0 \leq r < |b|$.

$a = 7, b = 3$ then $q = 2, r = 1$.

$a = 7, b = -3$ then $q = -2, r = 1$.

$a = -7, b = 3$ then $q = -3, r = 2$.

$a = -7, b = -3$ then $q = 3, r = 2$.

But, in \texttt{python} (using \texttt{//} for floor divison):

$a = 7, b = 3$ then $q = 2, r = 1$.

$a = 7, b = -3$ then $q = -3, r = -2$.

$a = -7, b = 3$ then $q = -3, r = 2$.

$a = -7, b = -3$ then $q = 2, r = -1$.

However, if we use truncated divison:

$a = 7, b = 3$ then $q = 2, r = 1$.

$a = 7, b = -3$ then $q = -2, r = 1$.

$a = -7, b = 3$ then $q = -2, r = -1$.

$a = -7, b = -3$ then $q = 2, r = -1$.

See: https://www.microsoft.com/en-us/research/publication/division-and-modulus-for-computer-scientists/
