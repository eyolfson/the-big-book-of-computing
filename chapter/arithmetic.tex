\chapter{Arithmetic}

Example of floor:

$\lfloor 0.8 \rfloor = 0$

Example of ceiling:

$\lceil 0.2 \rceil = 1$

Truncation:

\begin{equation*}
\mathrm{trunc}(x) = \begin{cases}
\lfloor x \rfloor &\text{if $x \geq 0$,}\\
\lceil x \rceil &\text{otherwise.}
\end{cases}
\end{equation*}

\section{Division}

Dividend / Divisor = (Quotient, Remainder)

$a / b = (q, r)$

where

$a = bq + r$

If you just want the remainder, the operation is $a \bmod b$.

there's a unique answer for $a, b$ when $0 \leq r < |b|$.

$a = 7, b = 3$ then $q = 2, r = 1$.

$a = 7, b = -3$ then $q = -2, r = 1$.

$a = -7, b = 3$ then $q = -3, r = 2$.

$a = -7, b = -3$ then $q = 3, r = 2$.

But, in \texttt{python} (using \texttt{//} for floor divison):

$a = 7, b = 3$ then $q = 2, r = 1$.

$a = 7, b = -3$ then $q = -3, r = -2$.

$a = -7, b = 3$ then $q = -3, r = 2$.

$a = -7, b = -3$ then $q = 2, r = -1$.

However, if we use truncated divison:

$a = 7, b = 3$ then $q = 2, r = 1$.

$a = 7, b = -3$ then $q = -2, r = 1$.

$a = -7, b = 3$ then $q = -2, r = -1$.

$a = -7, b = -3$ then $q = 2, r = -1$.

See: https://www.microsoft.com/en-us/research/publication/division-and-modulus-for-computer-scientists/

\subsection{Truncated}

$q = \textrm{trunc}(a/b)$

$r = a - b \textrm{trunc}(a/b)$

Same sign as the dividend (it should say unless it's zero!).

Case 1) If $a$ is negative.

Case 1a) If $b$ is positive then $r = a - b \lceil a/b \rceil$. Then $a \leq b
\lceil a/b \rceil$ because it rounds up (division is negative). \textbf{OR.}
Divide both sides by $b$ giving $a/b \leq \lceil a/b \rceil$, which is clearly
true.

Case 1b) If $b$ is negative then $r = a - b \lfloor a/b \rfloor$. Then $a \leq b
\lfloor a/b \rfloor$ because it rounds down (division is positive). So it's a
negative multiplied by a smaller positive number than without the floor,
resulting in a larger result. \textbf{OR.} Divide both sides by $b$ ($b$ is
negative so flip the inequality) giving $a/b \geq \lfloor a/b \rfloor$, which is
clearly true.

Case 2) If $a$ is positive.

Case 2a) If $b$ is positive then $r = a - b \lfloor a/b \rfloor$.
         We know $a \geq b \lfloor a/b \rfloor$.
         So $r$ is positive.

Case 2b) If $b$ is negative then $r = a - b \lceil a/b \rceil$.
         To be postive then $a \geq b \lceil a/b \rceil$.
         Divide both sides by $b$ giving $a \leq \lceil a/b \rceil$.
         So $r$ is positive.

\subsection{Floored}

\subsection{Euclidean}
