\chapter{Arithmetic}

Assuming a background with the following functions:

\begin{equation*}
  \text{floor}(x) = \lfloor x \rfloor = \max\{m \in \mathbb{Z} \mid m \leq x \}
\end{equation*}

\begin{equation*}
  \text{ceiling}(x) = \lceil x \rceil = \min\{n \in \mathbb{Z} \mid n \geq x \}
\end{equation*}

\begin{equation*}
\mathrm{trunc}(x) = \begin{cases}
\lfloor x \rfloor &\text{if $x \geq 0$,}\\
\lceil x \rceil &\text{otherwise.}
\end{cases}
\end{equation*}

\section{Division}

Every computer scientist and engineer needs to know the details of natural
number and integer division \cite{division-and-modulus-for-computer-scientists}.

Dividend / Divisor = (Quotient, Remainder)

$a / b = (q, r)$

where

$a = bq + r$

If you just want the remainder, the operation is $a \bmod b$.

there's a unique answer for $a, b$ when $0 \leq r < |b|$.

\begin{align*}
a &= 7  & b &= 3  & q &= 2  & r &= 1\\
a &= 7  & b &= -3 & q &= -2 & r &= 1\\
a &= -7 & b &= 3  & q &= -3 & r &= 2\\
a &= -7 & b &= -3 & q &= 3  & r &= 2
\end{align*}

But, in \texttt{python} (using \texttt{//} for floor divison):

\begin{align*}
a &= 7  & b &= 3  & q &= 2  & r &= 1\\
a &= 7  & b &= -3 & q &= -3 & r &= -2\\
a &= -7 & b &= 3  & q &= -3 & r &= 2\\
a &= -7 & b &= -3 & q &= 2  & r &= -1
\end{align*}

However, if we use truncated divison (\texttt{C/C++}):

\begin{align*}
a &= 7  & b &= 3  & q &= 2  & r &= 1\\
a &= 7  & b &= -3 & q &= -2 & r &= 1\\
a &= -7 & b &= 3  & q &= -2 & r &= -1\\
a &= -7 & b &= -3 & q &= 2  & r &= -1
\end{align*}

\subsection{Truncated}

In truncated division:

$q = \textrm{trunc}(\frac{a}{b})$

$r = a - b\ \textrm{trunc}(\frac{a}{b})$

Same sign as the dividend (it should say unless it's zero!).

Case 1)
If $a$ is negative.

Case 1a)
If $b$ is positive then $r = a - b \lceil \frac{a}{b} \rceil$.
Then $a \leq b \lceil \frac{a}{b} \rceil$ because it rounds up (division is
negative).
\textbf{OR.}
Divide both sides by $b$ giving $\frac{a}{b} \leq \lceil \frac{a}{b} \rceil$,
which is clearly true.

Case 1b)
If $b$ is negative then $r = a - b \lfloor \frac{a}{b} \rfloor$.
Then $a \leq b \lfloor \frac{a}{b} \rfloor$ because it rounds down (division is
positive).
So it's a negative multiplied by a smaller positive number than without the
floor, resulting in a larger result.
\textbf{OR.}
Divide both sides by $b$ ($b$ is negative so flip the inequality) giving
$\frac{a}{b} \geq \lfloor \frac{a}{b} \rfloor$, which is clearly true.

Case 2)
If $a$ is positive.

Case 2a)
If $b$ is positive then $r = a - b \lfloor \frac{a}{b} \rfloor$.
We know $a \geq b \lfloor \frac{a}{b} \rfloor$.
So $r$ is positive.

Case 2b)
If $b$ is negative then $r = a - b \lceil \frac{a}{b} \rceil$.
To be postive then $a \geq b \lceil \frac{a}{b} \rceil$.
Divide both sides by $b$ giving $a \leq \lceil \frac{a}{b} \rceil$.
So $r$ is positive.

\subsection{Floored}

In floored division:

$q = \lfloor \frac{a}{b} \rfloor$

$r = a - b \lfloor \frac{a}{b} \rfloor$

If $b$ is positive.

$\frac{a}{b} \geq \lfloor \frac{a}{b} \rfloor$

Result is positive.

If $b$ is negative.

$\frac{a}{b} \leq \lfloor \frac{a}{b} \rfloor$

Result is negative.

\subsection{Euclidean}

\subsection{Example}

\begin{lstlisting}
uint8_t mean_1(uint8_t x_0, uint8_t x_1)
{
  return (x_0 + x_1) / 2;
}
\end{lstlisting}

Wrong! Poor understanding of both \texttt{+} and \texttt{/}.

If $x_0 = 201$ and $x_1 = 203$, then $\text{mean}(x_0, x_1) = 202$.

Since all numbers are positive, it doesn't matter which div or mod functions we
use.
This code is the following:

\begin{equation*}
  \text{div}(\text{mod}(x_0 + x_1, 2^{8}), 2)
\end{equation*}

\begin{lstlisting}
uint8_t mean_2(uint8_t x_0, uint8_t x_1)
{
  return (x_0 / 2) + (x_1 / 2);
}
\end{lstlisting}

Also wrong. Probably even worse, it gives the wrong result whenever both numbers
are odd. The mod function will never not equal $a$ (or no overflow).

\begin{equation*}
  \text{div}(x_0 , 2) + \text{div}(x_1 , 2) \neq \text{div}(x_0 + x_1, 2)
\end{equation*}

\begin{equation*}
  \lfloor \frac{x_0}{2} \rfloor + \lfloor \frac{x_1}{2} \rfloor
  \neq \lfloor \frac{x_0 + x_1}{2} \rfloor
\end{equation*}

\begin{equation*}
  \lfloor \frac{x_0}{2} \rfloor + \lfloor \frac{x_1}{2} \rfloor
  + \lfloor \frac{\text{mod}(x_0, 2) + \text{mod}(x_1, 2)}{2} \rfloor
  = \lfloor \frac{x_0 + x_1}{2} \rfloor
\end{equation*}

Performance wise \texttt{mean\_2} is also 2x faster than \texttt{mean\_1}
(why?). The correct answer \texttt{mean\_3} (TODO) is approximately the same as
\texttt{mean\_2}.
