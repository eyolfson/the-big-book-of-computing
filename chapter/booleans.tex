\chapter{Booleans}

\section{Algebra}

\subsection{Basics}

There are 3 basic operations: \textbf{and}, \textbf{or}, and \textbf{not}. These
3 operations may go by their fancier names: \textbf{conjunction},
\textbf{disjunction}, and \textbf{negation}. The mathematical shorthand for
these 3 operations are: $\land$, $\lor$, and $\lnot$.

The first 2 operations take two booleans and produce one boolean as a result
(similar to addition for numbers). Since the value of a boolean can only be 0 or
1, we can enumerate all the different possibilities for these operations.  This
is shown in a truth table below:

\begin{center}
  \begin{tabular}{r r | r | r}
    $x$ & $y$ & $\land$ & $\lor$ \\
    \hline
    0 &   0 &       0 &      0 \\
    0 &   1 &       0 &      1 \\
    1 &   0 &       0 &      1 \\
    1 &   1 &       1 &      1 \\
  \end{tabular}
\end{center}

The last operation, \textbf{not}, takes one boolean value and changes its
value. So, $\lnot 0 = 1$ and $\lnot 1 = 0$.

\subsection{Derived}

There are 3 dervied operations: \textbf{equivalence}, \textbf{exclusive or
(xor)}, and \textbf{material implication}. The mathematical shorthand for these
3 operations are: $\equiv$, $\oplus$, and $\to$. Their truth table is shown
below:

\begin{center}
  \begin{tabular}{r r | r | r | r}
    $x$ & $y$ & $\equiv$ & $\oplus$ & $\to$ \\
    \hline
    0 &   0 &        1 &        0 &     1 \\
    0 &   1 &        0 &        1 &     0 \\
    1 &   0 &        0 &        1 &     1 \\
    1 &   1 &        1 &        0 &     1 \\
  \end{tabular}
\end{center}
